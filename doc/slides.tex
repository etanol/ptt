\documentclass{beamer}

\usepackage[spanish]{babel}
\usepackage[utf8]{inputenc}

% Beamer theme selection and other tuning
\usetheme{Boadilla}
\setbeamertemplate{footline}[page number]
\setbeamertemplate{navigation symbols}{}
%\setbeamertemplate{frametitle continuation}%
%    [from second][(\insertcontinuationcountroman)]

\title{POSIX Threads Tracing}
\subtitle{The gory details}
\author{Isaac Jurado}
\date{EITM}

\begin{document}

\maketitle

\begin{frame}{Design rationale}
  \begin{itemize}
    \item Mostly inspired by Cell/SMP Superscalar
    \item Play with compiler and linker magic.
    \item Objectives:
    \begin{itemize}
      \item Reduce source code modification
      \item Ease compilation process
      \item Simplify the API
    \end{itemize}
  \end{itemize}
\end{frame}

\begin{frame}{Automatic stuff}
  \begin{itemize}
    \item Initialization and finalization by using the \emph{constructor} and
    \emph{destructor} function attributes
    \item On demand thread buffer creation by intercepting the
    \texttt{pthread\_create} call at link time (details follow)
    \item Flushing to a temporary, per thread, file when the event buffer is
    full
    \item Postprocess at the end of the execution, directly generating
    \emph{Paraver}\textregistered{} traces
  \end{itemize}
\end{frame}

\begin{frame}{The linker trick}
  \begin{itemize}
    \item The \texttt{--wrap} option allows symbol redirection at link time.
    \item Using \texttt{--wrap symbol} behaves like the following:
    \begin{enumerate}
      \item Undefined references to \texttt{symbol} are linked with
      \texttt{\_\_wrap\_symbol} definitions
      \item Undefined references to \texttt{\_\_real\_symbol} are linked with
      \texttt{symbol} definitions
    \end{enumerate}
    \item This behaviour applies to all input objects
  \end{itemize}
\end{frame}

\begin{frame}{Partial traces and post-processing}
  \begin{itemize}
    \item Each trace flushes to its own temporary file
    \item Temporary files generated in \emph{/tmp} hoping the system
    has a competent administrator
    \item Each thread trace in binary format to reduce flush time:
    \begin{itemize}
      \item A single event uses 16 bytes:
      \begin{itemize}
        \item 8 byte timestamp
        \item 4 byte event type
        \item 4 byte event value
      \end{itemize}
      \item Memory buffer size specified in number of events
    \end{itemize}
    \item Upon finalization, all thread traces are merged in ascending timestamp
    order
      \begin{itemize}
        \item Each thread trace is naturally sorted
      \end{itemize}
    \item ROW file automatically generated
    \item PCF file automatically generated and customizable
  \end{itemize}
\end{frame}

\begin{frame}{Final balance}
  \begin{columns}[t]
    \column{.49\textwidth}
    Pros:
    \begin{itemize}
      \item Very simple to use, yet flexible enough
      \item Small code base, easy to understand and hack
      \begin{itemize}
        \item $\approx 560$ effective lines of code, including scripts and build
        system
        \item Right now only \texttt{pthread\_create} is intercepted, but adding
        more wrappers is very easy
      \end{itemize}
      \item Binary is instrumented in place, no load time dependencies
      \item A friendly build environment, \emph{autocrap} free, is provided
    \end{itemize}
    \column{.49\textwidth}
    Cons:
    \begin{itemize}
      \item Not so flexible, communication event API could be interesting
      \item So much automation sometimes gets in the way
      \begin{itemize}
        \item The main thread execution is traced completely
      \end{itemize}
      \item Load time interception may enable more configuration possibilities
      without recompilation (but not too many)
    \end{itemize}
  \end{columns}
\end{frame}

\end{document}
% vim:ft=tex
