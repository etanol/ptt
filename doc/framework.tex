\section{Framework implementation}
\label{sec:framework}

Once presented the mode of operation from the user point of view, it is time to
dip into the details.  As already mentioned in the introduction, the PTT
framework basically consists of two main components:

\begin{itemize}
\item Build system.
\item Tracing library.
\end{itemize}


\subsection{Build system}

To offer a functionality similar to \emph{automake} the build rules make use of
some advanced features from GNU Make.  Dependencies are properly defined so a
change in the tracing library or one of the preprocessing scripts.  However, the
dependency relationships established between the header and source files of the
user's programs are not as fine grained as they could.  A functionality that was
dropped due to the attached complexity increase to the build system.

Most of the magic involved in the automation explained in section
\ref{sec:pcf_sample} is achieved via simple preprocessing scripts that parse all
the PCF contents concatenated in a single stream.  For each program containing
custom PCF definitions, two files are generated:

\begin{itemize}
\item \verb:pcf_program.h:: containing the symbolic constants for the event
types and values.  This file is also included automatically, so no explicit
inclusion is required.
\item \verb:pcf_program.c:: containing the full PCF contents, in the form of a C
string, to be generated for each execution.
\end{itemize}

In case on extra PCF content is attached to the program, the header file is
not generated.

The build system also takes care of building the library in case it is
necessary, issuing the proper compiler and linker flags and embedding the
tracing library into the program binary.  In particular, one linker flag is of
special interest and it is discussed later in this section.


\subsection{Tracing library}

Using some compiler and linker special trickery allows the library to have most
of its code called automatically, freeing the user from that responsibility.
This approach has two main advantages:

\begin{itemize}
\item Eases the instrumentation process.
\item Reduces the troubleshooting scenarios produced by potential human errors,
like, for example, forgetting to insert a concrete initialization or
finalization function.
\end{itemize}

The automation happens at two levels: process and thread.  Each level needs
initialisation and finalisation.  At the process level, the library makes use of
the special GCC function attributes \emph{constructor} and \emph{destructor}.
They instruct the compiler to add a call before the execution of the main
function, and before finishing the process respectively.

At the thread level, detecting thread finalization is easy thanks to the
destructor function that can be associated to a \verb:pthread_key_t:.  This data
type provides \emph{thread local storage} which, at the same time, it is used by
the library to store separate per thread event buffers.  When the thread
finishes, all its defined \emph{thread local storage} keys are released, also
calling their destructor function, if defined.

Detecting thread creation is a different story.  It is important to
automatically intercept thread creation because the library needs to create
tracing structures for the new thread, otherwise the user would have to do it.
In this case some linker aid is necessary.  The idea is to intercept calls to
the \verb:pthread_create: function to be able to alter some of its parameters;
in particular the thread function and its parameter.  Essentially, once
\verb:pthread_create: is intercepted, the process of creating a thread is the
following:

\begin{enumerate}
\item User calls \verb:pthread_create: providing a function.
\item The call is redirected to the tracing library.
\item The tracing library allocates space for the new tracing structures.
\item The tracing library calls the real \verb:pthread_create: providing a
different function, an internal \emph{proxy} function.
\item The system creates the new thread and calls the proxy function.
\item The proxy function initializes the tracing structures.
\item The proxy function finally calls the user function provided initially.
\end{enumerate}

Focusing now on bare function call interception, there are two main approaches
that do not involve modifying the source code:

\begin{itemize}
\item The \verb:LD_PRELOAD: trick.
\item The linker \verb:--wrap: flag.
\end{itemize}

The first one is a run time solution, which has the advantage of not requiring
recompilation of the program.  It has the downside of requiring a slightly
complex environment to make it work.  Besides, the events are inserted in the
source code so recompilation is still necessary.  This leaves the second
solution.

The GNU linker provides a special \verb:--wrap: flag that allows to modify
external symbol resolution between objects.  The \verb:--wrap: switch requires
an argument: the symbol to catch.  When the linker finds an unresolved reference
to the symbol, e.g. \verb:func:, the reference is redirected to the symbol
\verb:__wrap_func: instead.  Then the linker looks for unresolved references to
the symbol \verb:__real_func:, which is redirected to \verb:func:.

Therefore, with a \verb:__wrap_pthread_create: function defined and the proper
command line switches, the linker does the job and the final binary does not
need any special environment nor library available.


\subsection{Post-processing}

To reduce memory consumption, the event buffers have a limited capacity.  When
buffers are filled they need to be flushed to disk to allow more events to enter
the buffer.  Each thread flushes to its own file to avoid locking, so at the end
of the execution there are as many files as threads have been executed.

At this point the traces need to be merged and each event needs to be adjusted a
bit.  In particular, the time of each event needs to be converted to
nanoseconds.  This is where post processing comes into play.

Inspired by the \emph{Cell Superscalar} framework.  Such post processing stage
is performed within the same process, right at the end of the execution.  This
way the post process can be simplified.

To minimize flush overhead, threads dump their buffers in raw binary form.  If
trace merging was separated from generation, byte endianness should be taken
into account as it would open the possibility of trying to post process binary
traces generated in a different architecture.

On the other hand, binary traces can be considered temporary files and created
as such; which raises the chances to be stored in a local or RAM file system.

Finally, some global information can be kept in main memory, saving the post
processor from having to parse it.


\subsection{Framework limitations}

In PTT, the main and only tracing primitive is the event.  There is no
communication nor state primitives as in other tracing tools.  Neither is the
availability of hardware counters nor, obviously, automatic events added with
such information.

Nevertheless, these limitations are in harmony with the main design goals of
simplicity and ease of use.

% vim:ft=tex:spell
